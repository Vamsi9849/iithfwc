% #######################################
% ########### FILL THESE IN #############
% #######################################
\def\mytitle{BCD to 2'S Complement}
\def\mykeywords{}
\def\myauthor{VAMSI SUNKARI}
\def\contact{vamsisunkari9849@gmail.com}
\def\mymodule{ Future Wireless Communication(FWC22040)}
% #######################################
% #### YOU DON'T NEED TO TOUCH BELOW ####
% #######################################
\documentclass[10pt, a4paper]{article}
\usepackage[a4paper,outer=1.5cm,inner=1.5cm,top=1.75cm,bottom=1.5cm]{geometry}
\twocolumn
\usepackage{graphicx}
\graphicspath{{./images/}}
%colour our links, remove weird boxes
\usepackage[colorlinks,linkcolor={black},citecolor={blue!80!black},urlcolor={blue!80!black}]{hyperref}
%Stop indentation on new paragraphs
\usepackage[parfill]{parskip}
%% Arial-like font
\usepackage{lmodern}
\renewcommand*\familydefault{\sfdefault}
%Napier logo top right
\usepackage{watermark}
%Lorem Ipusm dolor please don't leave any in you final report ;)
\usepackage{karnaugh-map} 
\usepackage{tabularx}
\usepackage{lipsum}
\usepackage{xcolor}
\usepackage{listings}
%give us the Capital H that we all know and love
\usepackage{float}
%tone down the line spacing after section titles
\usepackage{titlesec}
%Cool maths printing
\usepackage{amsmath}
%PseudoCode
\usepackage{algorithm2e}

\titlespacing{\subsection}{0pt}{\parskip}{-3pt}
\titlespacing{\subsubsection}{0pt}{\parskip}{-\parskip}
\titlespacing{\paragraph}{0pt}{\parskip}{\parskip}
\newcommand{\figuremacro}[5]{
    \begin{figure}[#1]
        \centering
        \includegraphics[width=#5\columnwidth]{#2}
        \caption[#3]{\textbf{#3}#4}
        \label{fig:#2}
    \end{figure}
}


 \lstset{
frame=single, 
breaklines=true,
columns=fullflexible
}

\thiswatermark{\centering \put(1,-110){\includegraphics[scale=0.07]{IIT_logo.png}} }
\title{\mytitle}
\author{\myauthor\hspace{1em}\\\contact\\IITH\hspace{0.5em}-\hspace{0.5em}\mymodule}
\date{}
\hypersetup{pdfauthor=\myauthor,pdftitle=\mytitle,pdfkeywords=\mykeywords}
\sloppy
% #######################################
% ########### START FROM HERE ###########
% #######################################
\begin{document}
 \maketitle
 \begin{abstract}
     %Replace the lipsum command with actual text 
  This objective of this document is to show Conversion between BCD to 2's Complement 
 \end{abstract}
    
 

 
    
    
    
 
 \section{Components}
 
     \begin{tabularx}{0.4\textwidth} {  
  | >{\centering\arraybackslash}X  
  | >{\centering\arraybackslash}X  
  | >{\centering\arraybackslash}X |}
  \hline
\textbf{Component} &  \textbf{Value} & \textbf{Quantity}\\
\hline
Arduino & UNO & 1 \\  
\hline
Resistor& 220ohm & 4\\ 
\hline
Bread board & - & 1 \\
\hline
Jumber wires & M-M & 20\\
\hline
Led & - & 4\\
\hline
\end{tabularx}

\section{BCD to 2's Complement}
 
 The Converter takes 4-bit boolean number as input
 and produces 2's complement as output.The corresponding truth tables is avaiable
 
     \begin{tabularx}{0.4\textwidth} {  
  | >{\centering\arraybackslash}X  
  | >{\centering\arraybackslash}X  
  | >{\centering\arraybackslash}X  
  | >{\centering\arraybackslash}X
  | >{\centering\arraybackslash}X 
  | >{\centering\arraybackslash}X 
  | >{\centering\arraybackslash}X 
  | >{\centering\arraybackslash}X |} 
  \hline 
  Z & Y & X & W & A & B & C & D \\ 
  \hline 
  0 & 0 & 0 & 0 & 0 & 0 & 0 & 0  \\ 
  \hline 
  0 & 0 & 0 & 1 & 1 & 1 & 1 & 1  \\ 
   \hline 
  0 & 0 & 1 & 0 & 1 & 1 & 1 & 0  \\ 
   \hline 
  0 & 0 & 1 & 1 & 1 & 1 & 0 & 1  \\ 
   \hline 
  0 & 1 & 0 & 0 & 1 & 1 & 0 & 0  \\ 
   \hline 
  0 & 1 & 0 & 1 & 1 & 0 & 1 & 1  \\ 
   \hline 
  0 & 1 & 1 & 0 & 1 & 0 & 1 & 0  \\ 
   \hline 
  0 & 1 & 1 & 1 & 1 & 0 & 0 &1  \\ 
   \hline 
  1 & 0 & 0 & 0 & 1 & 0 & 0 & 0  \\ 
   \hline 
  1 & 0 & 0 & 1 & 0 & 1 & 1 & 1  \\
   \hline
   
  
  \end{tabularx}
    
 \section{Logic}
      2's complement is used to represent the signed binary numbers. For finding 2's complement of the binary number, we will first find the 1's complement of the binary number and then add 1 to the least significant bit of it.


For example let us consider the binary number 1010 
\\1's Complement of 1010 is 0101
\\To obtain 2'S Complement 1 should be added  to LSB of 0101. Thus we get the 2's Complement of 1010 as 0110 .
\\Similarly 2's Complement of every BCD number can be obtained as shown in the table. 

    \section{Assembly Code}
        \subsection{Configuring output}
\begin{lstlisting}
ldi r16,0b00111100 
out DDRD,r16   
\end{lstlisting}
Here 2,3,4,5 pins are being declared as Outputs in portD 
\begin{lstlisting}
ldi r17,0b00000000
out DDRB,r17
in r17,PINB  ;Reading data from PINB
\end{lstlisting}
Here 10,11,12,13 pins are being declared as Inputs in portB

 \begin{lstlisting}
com r20
inc r20
out PORTD,r20
 \end{lstlisting}
"COM" complements the input in r20.\\
"INC" adds one to the LSB of the complemented\\  
    input.The final 2's Complement result is stored in r20 register. From r20 register the output is transferred to PORTD       
 


     
    \section{Hardware Connection}


    
     \begin{tabularx}{0.4\textwidth} {  
  | >{\centering\arraybackslash}X
  | >{\centering\arraybackslash}X 
  | >{\centering\arraybackslash}X 
  | >{\centering\arraybackslash}X  
  | >{\centering\arraybackslash}X |}
  \hline
\textbf{Arduino} &  2  &  3  &  4  &  5\\
\hline
\textbf{Leds}    & led1 & led2 & led3 & led4\\
\hline
 \end{tabularx}\\

 \begin{center}
 Fig :4
   \end{center}




   Give the connections as per Table 4. For taking the inputs connect 5V of arduino to +ve line of bread board to consider it as logic 'HIGH'.connect GND pin of arduino to -ve line of bread board to consider it as logic 'LOW'.


    
    
 
 


  
  \section{Software}
  1.Connect the arduino to the computer
  \\2.Download the follwing code
  
  \begin{lstlisting}
https://github.com/Vamsi9849/iithfwc/blob/main/assembly/codes/comp.asm
  \end{lstlisting}
  3.The led will ON and OFF according to the given input 




    

    
 


\end{document}