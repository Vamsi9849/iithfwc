% #######################################
% ########### FILL THESE IN #############
% #######################################
\def\mytitle{BCD to 2'S Complement}
\def\mykeywords{}
\def\myauthor{VAMSI SUNKARI}
\def\contact{vamsisunkari9849@gmail.com}
\def\mymodule{ Future Wireless Communication(FWC22040)}
% #######################################
% #### YOU DON'T NEED TO TOUCH BELOW ####
% #######################################
\documentclass[10pt, a4paper]{article}
\usepackage[a4paper,outer=1.5cm,inner=1.5cm,top=1.75cm,bottom=1.5cm]{geometry}
\twocolumn
\usepackage{graphicx}
\graphicspath{{./images/}}
%colour our links, remove weird boxes
\usepackage[colorlinks,linkcolor={black},citecolor={blue!80!black},urlcolor={blue!80!black}]{hyperref}
%Stop indentation on new paragraphs
\usepackage[parfill]{parskip}
%% Arial-like font
\usepackage{lmodern}
\renewcommand*\familydefault{\sfdefault}
%Napier logo top right
\usepackage{watermark}
%Lorem Ipusm dolor please don't leave any in you final report ;)
\usepackage{karnaugh-map} 
\usepackage{tabularx}
\usepackage{lipsum}
\usepackage{xcolor}
\usepackage{listings}
%give us the Capital H that we all know and love
\usepackage{float}
%tone down the line spacing after section titles
\usepackage{titlesec}
%Cool maths printing
\usepackage{amsmath}
%PseudoCode
\usepackage{algorithm2e}

\titlespacing{\subsection}{0pt}{\parskip}{-3pt}
\titlespacing{\subsubsection}{0pt}{\parskip}{-\parskip}
\titlespacing{\paragraph}{0pt}{\parskip}{\parskip}
\newcommand{\figuremacro}[5]{
    \begin{figure}[#1]
        \centering
        \includegraphics[width=#5\columnwidth]{#2}
        \caption[#3]{\textbf{#3}#4}
        \label{fig:#2}
    \end{figure}
}


 \lstset{
frame=single, 
breaklines=true,
columns=fullflexible
}

\thiswatermark{\centering \put(1,-110){\includegraphics[scale=0.07]{IIT_logo.png}} }
\title{\mytitle}
\author{\myauthor\hspace{1em}\\\contact\\IITH\hspace{0.5em}-\hspace{0.5em}\mymodule}
\date{}
\hypersetup{pdfauthor=\myauthor,pdftitle=\mytitle,pdfkeywords=\mykeywords}
\sloppy
% #######################################
% ########### START FROM HERE ###########
% #######################################
\begin{document}
 \maketitle
 \begin{abstract}
     %Replace the lipsum command with actual text 
  This objective of this document is to show Conversion between BCD to 2's Complement 
 \end{abstract}
    
 

 
    
    
    
 
 \section{Components}
 
     \begin{tabularx}{0.4\textwidth} {  
  | >{\centering\arraybackslash}X  
  | >{\centering\arraybackslash}X  
  | >{\centering\arraybackslash}X |}
  \hline
\textbf{Component} &  \textbf{Value} & \textbf{Quantity}\\
\hline
Arduino & UNO & 1 \\  
\hline
Resistor& 220ohm & 4\\ 
\hline
Bread board & - & 1 \\
\hline
Jumber wires & M-M & 20\\
\hline
Led & - & 4\\
\hline
\end{tabularx}

\section{BCD to 2's Complement}
 
 The Converter takes 4-bit boolean number as input
 and produces 2's complement as output.The corresponding truth tables is avaiable
 
     \begin{tabularx}{0.4\textwidth} {  
  | >{\centering\arraybackslash}X  
  | >{\centering\arraybackslash}X  
  | >{\centering\arraybackslash}X  
  | >{\centering\arraybackslash}X
  | >{\centering\arraybackslash}X 
  | >{\centering\arraybackslash}X 
  | >{\centering\arraybackslash}X 
  | >{\centering\arraybackslash}X |} 
  \hline 
  Z & Y & X & W & A & B & C & D \\ 
  \hline 
  0 & 0 & 0 & 0 & 0 & 0 & 0 & 0  \\ 
  \hline 
  0 & 0 & 0 & 1 & 1 & 1 & 1 & 1  \\ 
   \hline 
  0 & 0 & 1 & 0 & 1 & 1 & 1 & 0  \\ 
   \hline 
  0 & 0 & 1 & 1 & 1 & 1 & 0 & 1  \\ 
   \hline 
  0 & 1 & 0 & 0 & 1 & 1 & 0 & 0  \\ 
   \hline 
  0 & 1 & 0 & 1 & 1 & 0 & 1 & 1  \\ 
   \hline 
  0 & 1 & 1 & 0 & 1 & 0 & 1 & 0  \\ 
   \hline 
  0 & 1 & 1 & 1 & 1 & 0 & 0 &1  \\ 
   \hline 
  1 & 0 & 0 & 0 & 1 & 0 & 0 & 0  \\ 
   \hline 
  1 & 0 & 0 & 1 & 0 & 1 & 1 & 1  \\
   \hline
   
  
  \end{tabularx}
    
 \section{Boolean logic}
    Using Boolean logic,output A in table 0 can be expressed in terms of inputs W,X,Y,Z as
    
    A=Z'W+Z'X+Z'YX'+ZY'X'W'
    
    B=Z'YX'W'+Y'X'W+Z'Y'X
    
    C=Z'X'W+Y'X'W+Z'XW'
    
    D=Z'W+Y'X'W

    \section{karnaugh-map}
    \begin{center}
 \begin{karnaugh-map}[4][4][1][$XW$][$ZY$] 
    \minterms{1,2,3,4,5,6,7,8} 
    \maxterms{0,9,10,11,12,13,14,15} 
    \implicant{1}{7} 
    \implicant{4}{5} 
    \implicant{3}{2}
    \implicant{8}{8}
    \end{karnaugh-map}
    \end{center}
    \begin{center}
          FIG 1: K-map for A
    \end{center}
    \vspace{20mm}
    \begin{center}
 \begin{karnaugh-map}[4][4][1][$XW$][$ZY$] 
    \minterms{1,2,3,4,9} 
    \maxterms{0,5,6,7,8,10,11,12,13,14,15} 
    \implicantedge{1}{1}{9}{9} 
    \implicant{3}{2} 
    \implicant{4}{4}
    \end{karnaugh-map}
    \end{center}
    \begin{center}
        Fig 2: K-map for B 
    \end{center}
    \newpage
    
    \begin{center}
    \begin{karnaugh-map}[4][4][1][$XW$][$ZY$] 
    \minterms{1,2,5,6,9} 
    \maxterms{0,3,4,7,8,10,11,12,13,14,15} 
    \implicantedge{1}{1}{9}{9} 
    \implicant{1}{5} 
    \implicant{2}{6}
    \end{karnaugh-map}
    \end{center}
    \begin{center}
        Fig 3: K-map for C    
    \end{center}
    
   \begin{center}
    \begin{karnaugh-map}[4][4][1][$XW$][$ZY$] 
    \minterms{1,3,5,7,9} 
    \maxterms{0,2,4,6,8,10,11,12,13,14,15} 
    \implicantedge{1}{1}{9}{9} 
    \implicant{1}{7} 
    \end{karnaugh-map}
    \end{center}
    
    \begin{center}
        Fig 4: K-map for D
    \end{center}

     
    \section{Hardware Connection}


    
     \begin{tabularx}{0.4\textwidth} {  
  | >{\centering\arraybackslash}X
  | >{\centering\arraybackslash}X 
  | >{\centering\arraybackslash}X 
  | >{\centering\arraybackslash}X  
  | >{\centering\arraybackslash}X |}
  \hline
\textbf{Arduino} &  2  &  3  &  4  &  5\\
\hline
\textbf{Leds}    & led1 & led2 & led3 & led4\\
\hline
 \end{tabularx}\\

 \begin{center}
 Fig :4
   \end{center}




   Give the connections as per Table 4. For taking the inputs connect 5V of arduino to +ve line of bread board to consider it as logic 'HIGH'.connect GND pin of arduino to -ve line of bread board to consider it as logic 'LOW'.


    
    
 
 


  
  \section{Software}
  1.Connect the arduino to the computer
  \\2.Download the follwing code
  
  \begin{lstlisting}
https://github.com/Vamsi9849/iithfwc/blob/main/assignment%20codes/codes/bcd%20to%202's%20complement.txt
  \end{lstlisting}
  3.The led will ON and OFF according to the given input 




    

    
 


\end{document}


